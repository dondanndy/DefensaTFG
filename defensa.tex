\documentclass[xcolor=table]{beamer}
\usepackage[T1]{fontenc}
\usepackage[utf8]{inputenc}
\DeclareUnicodeCharacter{2212}{-}
\usepackage[spanish, es-tabla]{babel}
\usepackage{mathtools}
\usepackage{amsthm}
\usepackage{amsmath}
\usepackage{amsfonts}
\usepackage{amssymb}
\usepackage{makeidx}
\usepackage{graphicx}
% \usepackage[table,xcdraw]{xcolor}
\usepackage{lmodern}
\usepackage{tikz}
\usepackage{pgfplots}
\pgfplotsset{compat=newest}
\usepackage[font=small,justification=centering,labelformat=empty]{caption}
\usepackage[labelformat=empty]{subcaption}
\usetikzlibrary{arrows,shapes,positioning,calc,babel,pgfplots.groupplots, pgfplots.statistics}
\usepackage{pgfkeys} % LATEX
\usepackage{pgffor}
\usepackage{adjustbox}
\usepackage{bm}
\usepackage{siunitx}
\usepackage{multirow}
% \usepackage{multicol}
% \usepackage{afterpage}
\usepackage{xcolor}
\usepackage{soul}
\usepackage{float}
\usepackage[italicdiff]{physics}


\usetheme{metropolis}           % Use metropolis theme
\setbeamercolor{background canvas}{fg=black,bg=white}
\setbeamertemplate{frame footer}{\insertsection}
\title{Estudio del posicionamiento de un robot móvil mediante sensores "Ultra-wide Band"}
\subtitle{Trabajo de Fin de Grado \\ Grado en Física}
\date{24 de septiembre de 2020}
\author{Daniel A. Durán García \\ Tutor: Carlos Javier García Orellana}
\institute{Universidad de Extremadura}
\logo{\includegraphics[height=2cm]{pic/uex_logo_3.png}}


\begin{document}
  % \maketitle
  % \begin{frame}

  %   \title[About Beamer] %optional
  %   {About the Beamer class in presentation making}

  %   \subtitle{A short story}
  % \end{frame}

  \frame{\titlepage}
  \logo{}
  
  \begin{frame}{Índice}
    \begin{enumerate}
      \item Posicionamiento local
      \item Técnicas de posicionamiento
      \item Metodología
      \item Resultados
      \begin{itemize}
        \item Laboratorio
        \item Edificio de Física
      \end{itemize}
      \item Conclusiones
    \end{enumerate}
    \end{frame}


  \section{Posicionamiento local}
    \begin{frame}{Posicionamiento local}
      En interiores es imposible utilizar tecnologías de posicionamiento global basadas en señal de satélites (GPS).

      Existen tecnologías alternativas con mayor precisión destinadas a su uso en estos entornos interiores.
      
      \vspace*{-0.4cm}
      \begin{figure}[H]
        \centering
        \def\svgwidth{1.25\linewidth}
        \hspace*{-0.2cm}
        \input{./fig/general.pdf_tex}
      % \caption{Posicionamiento por tiempo de vuelo (TOA).}
        \label{fig:general}
      \end{figure}
    \end{frame}

  \section{Técnicas de posicionamiento}

    \begin{frame}{Técnicas de posicionamiento}
    \begin{itemize}
      \item \textit{Fingerprinting}
      \item Algoritmos geométricos
      % Machine Learning
      \begin{itemize}
        \item Tiempo de vuelo (TOA)
        \item Diferencia de tiempo de vuelo (TDOA)
        \item Potencia percibida (RSSI)
        \item Ángulo de llegada (AOA)
      \end{itemize}
    \end{itemize}
    \end{frame}

    % \begin{frame}{Técnicas geométricas}
    % De forma general
    %   \begin{equation*}\label{eq:geom}
    %     \vb{r} = \vb{f}(x_1, x_2, \ldots, x_n) + \vb{n}
    %   \end{equation*}

    %   Límite teórico (Cota inferior de Cramer Rao):
    %   \begin{equation*}
    %     \sigma^2(\theta_i) \geq [\mathcal{I}(\bm{\theta})^{-1}]_{ii}
    %   \end{equation*}
    %   donde
    %   \begin{equation*}
    %     [\mathcal{I}(\bm{\theta})]_{ij} = \frac{1}{\sigma^2} \sum_{n=0}^{N=1} \pdv{s[n,\bm{\theta}]}{\theta_i}\pdv{s[n,\bm{\theta}]}{\theta_j}
    %   \end{equation*}
    % \end{frame}

    \begin{frame}{Tiempo de vuelo (TOA)}
      \begin{columns}
        \begin{column}{0.5\textwidth}
          Conociendo la velocidad de propagación, es posible determinar la distancia entre baliza y objetivo midiendo el tiempo de vuelo de la señal.

          \vspace{.5cm}
          Requiere la sincronización de relojes de balizas y objetivo.
        \end{column}
        \begin{column}{0.5\textwidth}  
          \begin{figure}[H]
            \centering
            \def\svgwidth{\linewidth}
            \input{./fig/TOA.pdf_tex}
          % \caption{Posicionamiento por tiempo de vuelo (TOA).}
            \label{fig:TOA}
        \end{figure}
        \end{column}
      \end{columns}
    \end{frame}

    % \begin{frame}{Tiempo de vuelo (TOA)}
    %   \begin{columns}
    %     \begin{column}{0.5\textwidth}
    %       Cota inferior:
    %       \begin{equation*}\label{eq:CRLB_TOA}
    %         \text{CRLB} = \sigma^2(\tau) \geq \frac{1}{8\pi^2\text{SNR}\beta^2}
    %     \end{equation*}
    %     donde
    %     \begin{equation*}
    %       \beta^2 = \frac{\int_{-\infty}^{\infty} \omega^2 \abs{S(\omega)}^2\dd{\omega}}{\int_{-\infty}^{\infty} \abs{S(\omega)}^2\dd{\omega}}
    %   \end{equation*}
    %     \end{column}
    %     \begin{column}{0.5\textwidth}  
    %       \begin{figure}[H]
    %         \centering
    %         \def\svgwidth{\linewidth}
    %         \input{./fig/TOA.pdf_tex}
    %       % \caption{Posicionamiento por tiempo de vuelo (TOA).}
    %         \label{fig:TOA}
    %     \end{figure}
    %     \end{column}
    %   \end{columns}
    % \end{frame}

    \begin{frame}{Diferencia de tiempo de vuelo (TDOA)}
      \begin{columns}
        \begin{column}{0.5\textwidth}
          Para evitar la sincronización de relojes con el objetivo, se toma una baliza como referencia y se mide la diferencia de tiempo entre la emisión de pulsos respecto las demás.

          \vspace{0.5cm}
          Misma cota inferior que en el tiempo de vuelo.
        \end{column}
        \begin{column}{0.5\textwidth}  
          \begin{figure}[H]
            \centering
            \def\svgwidth{\linewidth}
            \input{./fig/TDOA.pdf_tex}
            \label{fig:TDOA}
        \end{figure}
        \end{column}
        \end{columns}
    \end{frame}

    \begin{frame}{Potencia percibida (RSSI)}
      \begin{columns}
        \begin{column}{0.5\textwidth}
          Aprovechando el decaimiento de la señal, es posible determinar la distancia entre la baliza y el objetivo.

          \vspace{0.5cm}
          Muy susceptible a interferencias.
        \end{column}
        \begin{column}{0.5\textwidth}  
          \begin{figure}[H]
            \centering
            \def\svgwidth{\linewidth}
            \input{./fig/TOA.pdf_tex}
          % \caption{Posicionamiento por tiempo de vuelo (TOA).}
            \label{fig:TOA}
        \end{figure}
        \end{column}
        \end{columns}
    \end{frame}

    % \begin{frame}{Potencia percibida (RSSI)}
    %   \begin{columns}
    %     \begin{column}{0.5\textwidth}
    %       Ecuación general del decaimiento de la potencia de la señal:
    %       \begin{equation*}\label{eq:RSSI}
    %         p(r) = p(r_0) - 10n\log(\frac{r}{r_0})
    %       \end{equation*}

    %       Cota inferior:
    %       \begin{equation*}\label{eq:CRLB_RSSI}
    %         \sigma^2(d_i) = \frac{\ln(10)\sigma_{sh}d}{10n}
    %     \end{equation*}
    %     \end{column}
    %     \begin{column}{0.5\textwidth}  
    %       \begin{figure}[H]
    %         \centering
    %         \def\svgwidth{\linewidth}
    %         \input{./fig/TOA.pdf_tex}
    %       % \caption{Posicionamiento por tiempo de vuelo (TOA).}
    %         \label{fig:TOA}
    %     \end{figure}
    %     \end{column}
    %     \end{columns}
    % \end{frame}

    \begin{frame}{Ángulo de llegada (AOA)}
      \begin{columns}
        \begin{column}{0.5\textwidth}
          Para evitar cualquier tipo de sincronización de relojes es posible usar el ángulo de llegada de la señal al objetivo.

          \vspace{0.5cm}
          Requiere el uso de antenas direccionales.
        \end{column}
        \begin{column}{0.5\textwidth}  
          \begin{figure}[H]
            \centering
            \def\svgwidth{\linewidth}
            \input{./fig/AOA.pdf_tex}
          % \caption{Posicionamiento por tiempo de vuelo (TOA).}
            \label{fig:AOA}
        \end{figure}
        \end{column}
        \end{columns}
      \end{frame}

      % \begin{frame}{Ángulo de llegada (AOA)}
      %   \begin{columns}
      %     \begin{column}{0.5\textwidth}
      %       Cota inferior:
      %       \begin{equation*}
      %         \sigma^2(\phi) = \frac{c^2}{2\pi^2 \text{SNR} \beta^2 N(N^2-1)d\cos(\phi)}
      %       \end{equation*}
      %       con $N$ balizas.
      %     \end{column}
      %     \begin{column}{0.5\textwidth}  
      %       \begin{figure}[H]
      %         \centering
      %         \def\svgwidth{\linewidth}
      %         \input{./fig/AOA.pdf_tex}
      %       % \caption{Posicionamiento por tiempo de vuelo (TOA).}
      %         \label{fig:AOA}
      %     \end{figure}
      %     \end{column}
      %     \end{columns}
      % \end{frame}

  \begin{frame}{Tecnologías}
    \begin{itemize}
      \item Ultrasonido
      \item Luz
      \item Bluetooth
      \item Wi-Fi
      \item Ultra-Wide Band
    \end{itemize}
    % Precsion temporal menor imposibilita el uso de tecnicas geometricas.
  \end{frame}

  \section{Ultra-wide Band}
  % \begin{frame}{Ultra-wide Band}
  %   Emisión de radiofrecuencia con un ancho de banda de al menos 500 MHz o más del 20\% de la frecuencia central.
  % \end{frame}

  \begin{frame}{Ultra-wide Band}
    Emisión de radiofrecuencia con un ancho de banda de al menos 500 MHz o más del 20\% de la frecuencia central.
    
    \vspace{0.3cm}
    Ventajas
    \begin{itemize}
      \item El gran ancho de banda permite pulsos muy cortos en el tiempo, lo que permite una gran resolución temporal.
      \item Pocas probabilidades de interferencias. %Influencia
      \item Bajo consumo energético.
      \item Gran capacidad de transmisión de datos.
    \end{itemize}
  \end{frame}

\section{Metodología}

  \begin{frame}{Sensores UWB}
    \begin{columns}
      \begin{column}{0.5\textwidth}
        Se utilizó el kit comercial MDEK1001, de la empresa DecaWave

        \vspace{0.5cm}
        Realiza el posicionamiento con TOA y TDOA.

        \vspace{0.5cm}
        Anuncia una precisión de hasta 10cm.
      \end{column}
      \begin{column}{0.5\textwidth}  
        \begin{figure}[H]
          \centering
          \includegraphics[width=0.45\textwidth]{pic/sensor.jpg}
          % \caption{Módulo de sensor UWB DWM1001 dentro de la carcasa que lo contiene.}
          \label{fig:robot}
      \end{figure}
      \end{column}
    \end{columns}
  \end{frame}

  \begin{frame}{Robot}
    \begin{columns}
      \begin{column}{0.5\textwidth}
        El robot autónomo utilizado fue el Turtlebot 2.

        \vspace{0.5cm}
        Es posible su uso con el entorno ROS.
      \end{column}
      \begin{column}{0.5\textwidth}  
        \begin{figure}[H]
          \centering
          \includegraphics[width=0.65\textwidth]{pic/robot_fisica.jpg}
          % \caption{Módulo de sensor UWB DWM1001 dentro de la carcasa que lo contiene.}
          \label{fig:robot}
      \end{figure}
      \end{column}
    \end{columns}
  \end{frame}

  \begin{frame}{Robot}
    \begin{columns}
      \begin{column}{0.5\textwidth}
        ROS incluye un paquete de navegación autónoma del robot.

        \vspace{0.5cm}
        En un entorno controlado, permite enviar coordenadas a las que el robot se desplace, evitando obstáculos.
      \end{column}
      \begin{column}{0.5\textwidth}
        \begin{figure}[H]
          \centering
          \includegraphics[width=\textwidth]{pic/Trayectoria.png}
          % \caption{Módulo de sensor UWB DWM1001 dentro de la carcasa que lo contiene.}
          \label{fig:robot}
      \end{figure}
      \end{column}
    \end{columns}
  \end{frame}

  \begin{frame}{Toma de datos}
    Se eligieron dos escenarios para la toma de datos:
    \begin{itemize}
      \item Laboratorio de Robótica 0L3 del Instituto de Computación Científica Avanzada de la Universidad de Extremadura.
      \item Primera planta del edificio B de la Facultad de Física.
    \end{itemize}

    \vspace{0.5cm}
    Configurando uno de los sensores como objetivo se colocó encima del robot.

    \vspace{0.5cm}
    Se programaron trayectorias que recorrían los dos escenarios por el robot, recabando datos del posicionamiento local del robot y de los sensores de UWB.
  \end{frame}

  \begin{frame}{Toma de datos}
    Previamente a la toma de datos con el robot, se realizaron simulaciones de las trayectorias en el simulador Stage incluido en ROS.

    Para una navegación más precisa, se realizaron de forma previa los mapas de ambos escenarios usados en dicha simulación y en la navegación con el robot.
    % Hicimos los mapas.

    \begin{figure}[H]
      \centering
      \includegraphics[height=3.5cm]{pic/Stage-rviz.png}
      % \caption{Módulo de sensor UWB DWM1001 dentro de la carcasa que lo contiene.}
      \label{fig:robot}
  \end{figure}
  \end{frame}

\section{Resultados}
  \begin{frame}{Resultados}
    Se tomó el posicionamiento local del robot como \textit{ground truth}:
    \begin{equation*}\label{eq:Diff_eje}
      \text{Error eje} = \text{Valor posición robot} - \text{Valor posición sensores}
    \end{equation*}

    Se interpretó el error de posicionamiento como el módulo del vector de diferencia entre los vectores de posición de robot y sensores.
    \begin{figure}[H]
      \centering
      \input{graph/diff_pos.tex}
      % \caption{Definición del vector de error de posición.}
      % \label{fig:diff_pos}
    \end{figure}
  \end{frame}

\section[Laboratorio]{Resultados}

  \begin{frame}{Laboratorio}
    En el laboratorio se podía encontrar una superficie amplia sin obstáculos.
    \begin{figure}[H]
      \begin{subfigure}[b]{.45\textwidth}
        \centering
        \includegraphics[width=\textwidth]{pic/lab1.jpg}
        % \caption{Eje x}
        \label{fig:foto_lab1}
      \end{subfigure}
      \begin{subfigure}[b]{.45\textwidth}
        \centering
        \includegraphics[width=\textwidth]{pic/lab2.jpg}
        % \caption{Eje y}
        \label{fig:foto_lab2}
      \end{subfigure}
      \caption{Fotografías del laboratorio de Robótica del ICCAEx.}
      \label{fig:foto_lab}
    \end{figure}
  \end{frame}

  \begin{frame}{Laboratorio - Puntos a medir y trayectoria}
    % Comentar aqui las repeticiones de las trayectorias.
    \begin{figure}[H]
      \begin{subfigure}[b]{.3\textwidth}
        \centering
        \def\svgwidth{0.85\linewidth}
        \input{./fig/lab.pdf_tex} 
        \caption{Puntos a evaluar}
        \label{fig:puntos}
      \end{subfigure}
      \begin{subfigure}[b]{.3\textwidth}
        \centering
        \def\svgwidth{0.85\linewidth}
        \input{./fig/lab_espiral.pdf_tex} 
        \caption{Trayectoria en espiral}
        \label{fig:espiral}
      \end{subfigure}
      \begin{subfigure}[b]{.3\textwidth}
          \centering
          \def\svgwidth{0.85\linewidth}
        \input{./fig/lab_vertical.pdf_tex} 
          \caption{Trayectoria en vertical}
          \label{fig:vertical}
        \end{subfigure}
      % \caption{Puntos de medida del laboratorio y trayectorias a recorrer por el robot.}
      \label{fig:laboratorio}
      \end{figure}
  \end{frame}

  \begin{frame}{Laboratorio - Prueba con 4 balizas}
    % Configuración con 4 balizas 
    \begin{figure}[H]
      \centering
      \def\svgwidth{0.4\linewidth}
      \input{./fig/lab_4sensores.pdf_tex} 
      % \caption{4 balizas}
      \label{fig:lab_4sens}
    \end{figure}
  \end{frame}

  \begin{frame}{Laboratorio - Resultados con 4 balizas}
    \begin{figure}[H]
      \centering
      \scalebox{0.6}{\input{graph/res/4sensores.pgf}}
      % \caption{Resultados de las medidas del laboratorio con 6 balizas. Entre paréntesis se indica la desviación estándar de los errores obtenidos en cada punto.}
      \label{fig:res_4_lab}
  \end{figure}
  \end{frame}
  
  \begin{frame}{Laboratorio - Resultados con 4 balizas}
    \begin{table}[H]
      \tiny
      \hspace*{-0.5cm}
      \centering
      \begin{tabular}{c|c|c|c|c|c|}
      \cline{2-6}
                                       & \begin{tabular}[c]{@{}c@{}}Error en \\ eje $x$ (cm) \end{tabular} & \begin{tabular}[c]{@{}c@{}} Error en \\ eje $y$ (cm) \end{tabular} & \begin{tabular}[c]{@{}c@{}} Error en eje $x$ \\ (absoluto) (cm) \end{tabular} & \begin{tabular}[c]{@{}c@{}} Error en eje $y$ \\ (absoluto) (cm) \end{tabular} & \cellcolor[HTML]{C0C0C0} \begin{tabular}[c]{@{}c@{}} Error en \\ posición(cm) \end{tabular}\\ \hline
      \multicolumn{1}{|c|}{Media}      & -6.7  &  3.1  & 14.6 & 5.77 &  \cellcolor[HTML]{C0C0C0} 20.0   \\ \hline
      \multicolumn{1}{|c|}{Mediana}    & -3.3  &  4.1  & 13.1 & 4.9 &  \cellcolor[HTML]{C0C0C0} 17.7   \\ \hline
      \multicolumn{1}{|c|}{Desv. estándar} & 17.8  &  5.9  & 12.2 & 3.4 & \cellcolor[HTML]{C0C0C0} 9.6    \\ \hline
      \multicolumn{1}{|c|}{Máximo}     &  20.1 & 14.4  & 56.6 & 14.4 & \cellcolor[HTML]{C0C0C0} 57.6   \\ \hline
      \multicolumn{1}{|c|}{Mínimo}     & -56.6 & -12.4 & 0.5 & 0.7 & \cellcolor[HTML]{C0C0C0} 7.2    \\ \hline
      \end{tabular}
      \caption{Resumen de los errores en el laboratorio con 4 balizas.}
      \label{tab:resumen_lab_4}
    \end{table}
  \end{frame}

  \begin{frame}{Laboratorio - Prueba con 6 balizas}
    Configuración con 6 balizas
    \begin{figure}[H]
      \centering
      \def\svgwidth{0.4\linewidth}
      \input{./fig/lab_6sensores.pdf_tex} 
      % \caption{4 balizas}
      \label{fig:lab_6sens}
    \end{figure}
  \end{frame}

  \begin{frame}{Laboratorio - Resultados con 6 balizas}
    \begin{figure}[H]
      \centering
      \scalebox{0.6}{\input{graph/res/6sensores.pgf}}
      % \caption{Resultados de las medidas del laboratorio con 6 balizas. Entre paréntesis se indica la desviación estándar de los errores obtenidos en cada punto.}
      \label{fig:res_6_lab}
  \end{figure}
  \end{frame}

  \begin{frame}{Laboratorio - Resultados con 6 balizas}
    \begin{table}[H]
      \tiny
      \centering
      \hspace*{-0.5cm}
      \begin{tabular}{c|c|c|c|c|c|}
      \cline{2-6}
                                       & \begin{tabular}[c]{@{}c@{}}Error en \\ eje $x$ (cm) \end{tabular} & \begin{tabular}[c]{@{}c@{}} Error en \\ eje $y$ (cm) \end{tabular} & \begin{tabular}[c]{@{}c@{}} Error en eje $x$ \\ (absoluto) (cm) \end{tabular} & \begin{tabular}[c]{@{}c@{}} Error en eje $y$ \\ (absoluto) (cm) \end{tabular} & \cellcolor[HTML]{C0C0C0}  \begin{tabular}[c]{@{}c@{}} Error en \\ posición(cm) \end{tabular}\\ \hline
      \multicolumn{1}{|c|}{Media}      & -3.7  & 0.7  &  9.6 &  7.7  &  \cellcolor[HTML]{C0C0C0}  19.2   \\ \hline
      \multicolumn{1}{|c|}{Mediana}    & -2.2  & 2.08  &  8.8 &  6.0  &  \cellcolor[HTML]{C0C0C0}  18.2   \\ \hline
      \multicolumn{1}{|c|}{Desv. estándar} & 11.0  &  10.3  & 6.5 & 6.8 &  \cellcolor[HTML]{C0C0C0}  6.1    \\ \hline
      \multicolumn{1}{|c|}{Máximo}     & 15.9  & 17.4  & 24.9 & 33.9 & \cellcolor[HTML]{C0C0C0}  35.8   \\ \hline
      \multicolumn{1}{|c|}{Mínimo}     & -24.9 & -33.9 & 0.1 & 0.1 &  \cellcolor[HTML]{C0C0C0}  8.6    \\ \hline
      \end{tabular}
      \caption{Resumen de los errores en el laboratorio con 6 balizas.}
  \end{table}
  \end{frame}

  \begin{frame}{Laboratorio - Comparación de las configuraciones de balizas}
    % Quitar lo de a, b y c
    \begin{figure}[H]
      \centering
        \begin{subfigure}[b]{.3\textwidth}
          \centering
          \hspace*{-0.8cm}
          \input{./graph/boxplot_x.tex} 
          \vspace*{-0.5cm}
          \caption{Eje x}
          \label{fig:boxplot_lab_x}
        \end{subfigure}
        \hspace*{0.1cm}
        \begin{subfigure}[b]{.3\textwidth}
          \centering
          \input{./graph/boxplot_y.tex}  
          \vspace*{-0.5cm}
          \caption{Eje y}
          \label{fig:boxplot_lab_y}
        \end{subfigure}
        \hspace*{0.1cm}
        \begin{subfigure}[b]{.3\textwidth}
            \centering
            \input{./graph/boxplot_pos.tex}  
            \caption{Posición}
            \label{fig:boxplot_lab_pos}
          \end{subfigure}
        % \caption{Comparación del rendimiento de las configuraciones de balizas en el laboratorio.}
        \label{fig:boxplot_lab}
    \end{figure}
  \end{frame}
  

  \section[Edificio de Física]{Resultados}

  \begin{frame}{Edificio de Física}
    La primera planta del edificio de Física presenta una superficie mayor que el laboratorio con la presencia de obstáculos entre las balizas y el objetivo.

    \begin{figure}[H]
      \begin{subfigure}[b]{.45\textwidth}
        \centering
        \includegraphics[width=\textwidth]{pic/fisica1.jpg}
        % \caption{Eje x}
        \label{fig:foto_fisica1}
      \end{subfigure}
      \begin{subfigure}[b]{.45\textwidth}
        \centering
        \includegraphics[width=\textwidth]{pic/fisica2.jpg}
        % \caption{Eje y}
        \label{fig:foto_fisica2}
      \end{subfigure}
      \caption{Fotografías del edificio B de la Facultad de Física.}
      \label{fig:foto_fisica}
    \end{figure}
  \end{frame}

  \begin{frame}{Edificio de Física - Puntos a medir y trayectoria}
    \begin{figure}[H]
      \begin{subfigure}[b]{.42\textwidth}
        \centering
        \def\svgwidth{\linewidth}
        \input{./fig/fisica_num.pdf_tex} 
        \caption{Puntos a evaluar}
        \label{fig:puntos_fisica}
      \end{subfigure}
      ~~
      \begin{subfigure}[b]{.42\textwidth}
        \centering
        \def\svgwidth{\linewidth}
        \input{./fig/fisica_tray.pdf_tex} 
        \caption{Trayectoria}
        \label{fig:trayecto_fisica}
      \end{subfigure}
      % \caption{Disposición de la planta del edificio de Física.}
      \label{fig:fisica}
    \end{figure}
  \end{frame}

  \begin{frame}{Edificio de Física - Resultados con 4 balizas}
    Configuración de 4 balizas
    \begin{figure}[H]
      \centering
      \def\svgwidth{0.5\linewidth}
      \input{./fig/fisica_4sens.pdf_tex} 
      % \caption{4 balizas}
      \label{fig:sensores_fisica_4}
    \end{figure}
  \end{frame}
  
  \begin{frame}{Edificio de Física - Resultados con 4 balizas}
    \begin{figure}[H]
      \centering
      \scalebox{0.6}{\input{graph/res/fisica_4sensores.pgf}}
      \label{fig:res_fisica_4}
    \end{figure}
  \end{frame}
  
    \begin{frame}{Edificio de Física - Resultados con 4 balizas}
      \begin{table}[H]
        \tiny
        \hspace*{-0.5cm}
        \centering
        \begin{tabular}{c|c|c|c|c|c|}
        \cline{2-6}
                                                  & \begin{tabular}[c]{@{}c@{}}Error en \\ eje $x$ (cm) \end{tabular} & \begin{tabular}[c]{@{}c@{}} Error en \\ eje $y$ (cm) \end{tabular} & \begin{tabular}[c]{@{}c@{}} Error en eje $x$ \\ (absoluto) (cm) \end{tabular} & \begin{tabular}[c]{@{}c@{}} Error en eje $y$ \\ (absoluto) (cm) \end{tabular} & \cellcolor[HTML]{C0C0C0}  \begin{tabular}[c]{@{}c@{}} Error en \\ posición(cm) \end{tabular}\\ \hline
        \multicolumn{1}{|c|}{Media}               &   0.5   &   -8.6  &      15.3          &         15.4       &  \cellcolor[HTML]{C0C0C0}  26.1    \\ \hline
        \multicolumn{1}{|c|}{Mediana}             &   3.4   &   2.08  &      13.7          &         11.2       &  \cellcolor[HTML]{C0C0C0}  25.9    \\ \hline
        \multicolumn{1}{|c|}{Desv. estándar}      &   18.4  &   10.2  &      18.7          &         13.8       &  \cellcolor[HTML]{C0C0C0}  12.5    \\ \hline
        \multicolumn{1}{|c|}{Máximo}              &   27.8  &   39.9  &      40.8          &         49.6       &  \cellcolor[HTML]{C0C0C0}  52.3    \\ \hline
        \multicolumn{1}{|c|}{Mínimo}              &  -40.8  &   -49.6 &      2.3           &         0.5        &  \cellcolor[HTML]{C0C0C0}  5.5     \\ \hline
        \end{tabular}
        \caption{Resumen de los errores en el edificio de Física con 4 balizas.}
        \label{tab:media_fisica_4}
      \end{table}
    \end{frame}

  \begin{frame}{Edificio de Física - Resultados con 6 balizas}
    Configuración de 6 balizas
    \begin{figure}[H]
      \centering
      \def\svgwidth{0.5\linewidth}
      \input{./fig/fisica_6sens.pdf_tex} 
      % \caption{6 balizas}
      \label{fig:sensores_fisica_6}
    \end{figure}
  \end{frame}

  \begin{frame}{Edificio de Física - Resultados con 6 balizas}
    \begin{figure}[H]
      \centering
      \scalebox{0.6}{\input{graph/res/fisica_6sensores.pgf}}
      \label{fig:res_fisica_6}
  \end{figure}
  \end{frame}

  \begin{frame}{Edificio de Física - Resultados con 6 balizas}
    \begin{table}[H]
      \tiny
      \hspace*{-0.5cm}
      \centering
      \begin{tabular}{c|c|c|c|c|c|}
        \cline{2-6}
                                                  & \begin{tabular}[c]{@{}c@{}}Error en \\ eje $x$ (cm) \end{tabular} & \begin{tabular}[c]{@{}c@{}} Error en \\ eje $y$ (cm) \end{tabular} & \begin{tabular}[c]{@{}c@{}} Error en eje $x$ \\ (absoluto) (cm) \end{tabular} & \begin{tabular}[c]{@{}c@{}} Error en eje $y$ \\ (absoluto) (cm) \end{tabular} & \cellcolor[HTML]{C0C0C0}  \begin{tabular}[c]{@{}c@{}} Error en \\ posición(cm) \end{tabular}\\ \hline
        \multicolumn{1}{|c|}{Media}               &  -4.7   &   -11.9 &      16.0          &         12.1       &  \cellcolor[HTML]{C0C0C0}  23.7    \\ \hline
        \multicolumn{1}{|c|}{Mediana}             &  -3.8   &   -10.4 &      17.7          &         10.3       &  \cellcolor[HTML]{C0C0C0}  22.7    \\ \hline
        \multicolumn{1}{|c|}{Desv. estándar}      &   18.7  &   10.5  &      10.8          &         10.1       &  \cellcolor[HTML]{C0C0C0}  10.9    \\ \hline
        \multicolumn{1}{|c|}{Máximo}              &   32.1  &   1.1   &      40.8          &         30.8       &  \cellcolor[HTML]{C0C0C0}  44.1    \\ \hline
        \multicolumn{1}{|c|}{Mínimo}              &  -40.7  &   -30.9 &      3.3           &         0.4        &  \cellcolor[HTML]{C0C0C0}  7.8     \\ \hline
    \end{tabular}
    \caption{Resumen de los errores en el edificio de Física con 6 balizas.}
    \label{tab:media_fisica_6}
    \end{table}
  \end{frame}

  \begin{frame}{Edificio de Física - Resultados con 8 balizas}
    Configuración de 8 balizas
    \begin{figure}[H]
      \centering
      \def\svgwidth{0.5\linewidth}
      \input{./fig/fisica_8sens.pdf_tex} 
      % \caption{8 balizas}
      \label{fig:sensores_fisica_8}
    \end{figure}
  \end{frame}

  \begin{frame}{Edificio de Física - Resultados con 8 balizas}
    \begin{figure}[H]
      \centering
      \scalebox{0.6}{\input{graph/res/fisica_8sensores.pgf}}
      \label{fig:res_fisica_8}
  \end{figure}
  \end{frame}

  \begin{frame}{Edificio de Física - Prueba con 8 balizas}
    \begin{table}[H]
      \tiny
      \hspace*{-0.5cm}
      \centering
      \begin{tabular}{c|c|c|c|c|c|}
        \cline{2-6}
                                                  & \begin{tabular}[c]{@{}c@{}}Error en \\ eje $x$ (cm) \end{tabular} & \begin{tabular}[c]{@{}c@{}} Error en \\ eje $y$ (cm) \end{tabular} & \begin{tabular}[c]{@{}c@{}} Error en eje $x$ \\ (absoluto) (cm) \end{tabular} & \begin{tabular}[c]{@{}c@{}} Error en eje $y$ \\ (absoluto) (cm) \end{tabular} & \cellcolor[HTML]{C0C0C0}  \begin{tabular}[c]{@{}c@{}} Error en \\ posición(cm) \end{tabular}\\ \hline
        \multicolumn{1}{|c|}{Media}               &  -3.5   &  -6.2   &      11.9          &         13.7       &  \cellcolor[HTML]{C0C0C0}  22.3    \\ \hline
        \multicolumn{1}{|c|}{Mediana}             &  -4.7   &  -7.1   &      11.4          &         9.6        &  \cellcolor[HTML]{C0C0C0}  19.1    \\ \hline
        \multicolumn{1}{|c|}{Desv. estándar}      &  13.6   &   12.3  &      7.5           &         12.3       &  \cellcolor[HTML]{C0C0C0}  12.2    \\ \hline
        \multicolumn{1}{|c|}{Máximo}              &  21.0   &   48.9  &      25.8          &         48.9       &  \cellcolor[HTML]{C0C0C0}  55.5    \\ \hline
        \multicolumn{1}{|c|}{Mínimo}              &  -40.7  &  -30.5  &      3.3           &         1.5        &  \cellcolor[HTML]{C0C0C0}  5.6     \\ \hline
    \end{tabular}
    \caption{Resumen de los errores en el edificio de Física con 8 balizas.}
    \label{tab:media_fisica_8}
    \end{table}
  \end{frame}

  \begin{frame}{Edificio de Física - Comparación de resultados}
    \vspace*{-0.2cm}
    \begin{figure}[H]
      \centering
        \begin{subfigure}[b]{.3\textwidth}
          \centering
          \hspace*{-0.8cm}
          \input{./graph/boxplot_fisica_x.tex} 
          \vspace*{-0.5cm}
          \caption{Eje x}
          \label{fig:boxplot_fisica_x}
        \end{subfigure}
        \hspace*{0.1cm}
        \begin{subfigure}[b]{.3\textwidth}
          \centering
          \input{./graph/boxplot_fisica_y.tex}  
          \vspace*{-0.5cm}
          \caption{Eje y}
          \label{fig:boxplot_fisica_y}
        \end{subfigure}
        ~~~~
        \begin{subfigure}[b]{.3\textwidth}
            \centering
            \input{./graph/boxplot_fisica_pos.tex}  
            \vspace*{-0.5cm}
            \caption{Posición}
            \label{fig:boxplot_fisica_pos}
          \end{subfigure}
        % \caption{Comparación del rendimiento de las configuraciones de balizas en el edificio de Física.}
        \label{fig:boxplot_fisica}
    \end{figure}
  \end{frame}

\section{Conclusiones}
\begin{frame}{Conclusiones}
  % Pausita antes de acabar.

  Se ha medido el rendimiento del kit comercial DecaWave MDEK1001 de Ultra-wide Band con la ayuda de un robot autónomo en dos escenarios con distintas características.
  
  % En general, se obtienen errores dos órdenes de magnitud menores que la superficie total.

  Al usar solo 4 balizas en todo momento, la posición de las balizas es crítica para un correcto posicionamiento.

  La presencia de obstáculos que evitan la visión directa tienen un efecto crítico en el posicionamiento.

 Aunque la tecnología UWB ofrece el mejor rendimiento entre sus rivales, sus puntos débiles hacen que el problema del posicionamiento no esté del todo resuelto. 
\end{frame}

\logo{\includegraphics[height=2cm]{pic/uex_logo_3.png}}
\maketitle

\end{document}